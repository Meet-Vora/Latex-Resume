%%%%%%%%%%%%%%%%%
% This is an sample CV template created using altacv.cls
% (v1.6, 21 May 2021) written by LianTze Lim (liantze@gmail.com). Now compiles with pdfLaTeX, XeLaTeX and LuaLaTeX.
%
%% It may be distributed and/or modified under the
%% conditions of the LaTeX Project Public License, either version 1.3
%% of this license or (at your option) any later version.
%% The latest version of this license is in
%%    http://www.latex-project.org/lppl.txt
%% and version 1.3 or later is part of all distributions of LaTeX
%% version 2003/12/01 or later.t
%%%%%%%%%%%%%%%%

%% Use the "normalphoto" option if you want a normal photo instead of cropped to a circle
% \documentclass[10pt,a4paper,normalphoto]{altacv}

\documentclass[10pt,a4paper,ragged2e,withhyper]{altacv}
%% AltaCV uses the fontawesome5 and packages.
%% See http://texdoc.net/pkg/fontawesome5 for full list of symbols.

% Change the page layout if you need to
\geometry{left=1.25cm,right=1.25cm,top=1.5cm,bottom=1.5cm,columnsep=1.2cm}

% The paracol package lets you typeset columns of text in parallel
\usepackage{paracol}
\usepackage{hyperref}
\usepackage{fontawesome5}


% Change the font if you want to, depending on whether
% you're using pdflatex or xelatex/lualatex
\ifxetexorluatex
  % If using xelatex or lualatex:
  \setmainfont{Roboto Slab}
  \setsansfont{Lato}
  \renewcommand{\familydefault}{\sfdefault}
\else
  % If using pdflatex:
  \usepackage[rm]{roboto}
  \usepackage[defaultsans]{lato}
  % \usepackage{sourcesanspro}
  \renewcommand{\familydefault}{\sfdefault}
\fi

% Change the colours if you want to
\definecolor{SlateGrey}{HTML}{2E2E2E}
\definecolor{LightGrey}{HTML}{666666}
\definecolor{DarkPastelRed}{HTML}{450808}
\definecolor{PastelRed}{HTML}{8F0D0D}
\definecolor{GoldenEarth}{HTML}{E7D192}
% \colorlet{name}{black}
% \colorlet{tagline}{NewBlue}
% \colorlet{heading}{DarkPastelRed}
% % \colorlet{headingrule}{BDE5E1}
% \colorlet{subheading}{NewBlue}
% \colorlet{accent}{NewBlue}
% \colorlet{emphasis}{SlateGrey} % should be fine
% \colorlet{body}{LightGrey}

\definecolor{LightBlue}{HTML}{137dc9}
\definecolor{MidBlue}{HTML}{6BBED4}
\definecolor{DarkerBlue}{HTML}{2864cc}
% \definecolor
% \definecolor{}{}{}
\colorlet{name}{black}
\colorlet{tagline}{LightBlue}
\colorlet{heading}{DarkerBlue}
\colorlet{headingrule}{DarkerBlue}
\colorlet{subheading}{LightBlue}
\colorlet{accent}{LightBlue}
\colorlet{emphasis}{SlateGrey} % should be fine
\colorlet{body}{LightGrey}
% Change some fonts, if necessary
\renewcommand{\namefont}{\Huge\rmfamily\bfseries}
\renewcommand{\personalinfofont}{\footnotesize}
\renewcommand{\cvsectionfont}{\LARGE\rmfamily\bfseries}
\renewcommand{\cvsubsectionfont}{\large\bfseries}

% Change the bullets for itemize and rating marker
% for \cvskill if you want to
\renewcommand{\itemmarker}{{\small\textbullet}}
\renewcommand{\ratingmarker}{\faCircle}
%% Use (and optionally edit if necessary) this .tex if you
%% want to use an author-year reference style like APA(6)
%% for your publication list
% When using APA6 if you need more author names to be listed
% because you're e.g. the 12th author, add apamaxprtauth=12
\usepackage[backend=biber,style=apa6,sorting=ydnt]{biblatex}
\defbibheading{pubtype}{\cvsubsection{#1}}
\renewcommand{\bibsetup}{\vspace*{-\baselineskip}}
\AtEveryBibitem{\makebox[\bibhang][l]{\itemmarker}}
\setlength{\bibitemsep}{0.25\baselineskip}
\setlength{\bibhang}{1.25em}


%% Use (and optionally edit if necessary) this .tex if you
%% want an originally numerical reference style like IEEE
%% for your publication list
% \usepackage[backend=biber,style=ieee,sorting=ydnt]{biblatex}
%% For removing numbering entirely when using a numeric style
\setlength{\bibhang}{1.25em}
\DeclareFieldFormat{labelnumberwidth}{\makebox[\bibhang][l]{\itemmarker}}
\setlength{\biblabelsep}{0pt}
\defbibheading{pubtype}{\cvsubsection{#1}}
\renewcommand{\bibsetup}{\vspace*{-\baselineskip}}


%% sample.bib contains your publications
\addbibresource{sample.bib}
\begin{document}
\name{Meet Vora}
\tagline{Computer Science @ UC Berkeley | Backend-oriented Fullstack SWE}
%% You can add multiple photos on the left or right
% \photoR{2.8cm}{Globe_High}
% \photoL{2.5cm}{Yacht_High,Suitcase_High}

\personalinfo{%
  % Not all of these are required!
  \email{meetrvora1@gmail.com}
  \phone{(408) 768-7419}
%   \mailaddress{Åddrésş, Street, 00000 Cóuntry}
%   \location{Location, COUNTRY}
%   \homepage{www.homepage.com}
%   \twitter{@twitterhandle}
  \linkedin{linkedin.com/in/mrvora}
  \github{github.com/meet-vora}
%   \orcid{0000-0000-0000-0000}
  %% You can add your own arbitrary detail with
  %% \printinfo{symbol}{detail}[optional hyperlink prefix]
  % \printinfo{\faPaw}{Hey ho!}[https://example.com/]
  %% Or you can declare your own field with
  %% \NewInfoFiled{fieldname}{symbol}[optional hyperlink prefix] and use it:
  % \NewInfoField{gitlab}{\faGitlab}[https://gitlab.com/]
  % \gitlab{your_id}
  %%
  %% For services and platforms like Mastodon where there isn't a
  %% straightforward relation between the user ID/nickname and the hyperlink,
  %% you can use \printinfo directly e.g.
  % \printinfo{\faMastodon}{@username@instace}[https://instance.url/@username]
  %% But if you absolutely want to create new dedicated info fields for
  %% such platforms, then use \NewInfoField* with a star:
  % \NewInfoField*{mastodon}{\faMastodon}
  %% then you can use \mastodon, with TWO arguments where the 2nd argument is
  %% the full hyperlink.
  % \mastodon{@username@instance}{https://instance.url/@username}
}

\makecvheader
%% Depending on your tastes, you may want to make fonts of itemize environments slightly smaller
% \AtBeginEnvironment{itemize}{\small}

%% Set the left/right column width ratio to 6:4.
\columnratio{0.6}

% Start a 2-column paracol. Both the left and right columns will automatically
% break across pages if things get too long.
\begin{paracol}{2}
\cvsection{Experience}

\cvevent{Fullstack Software Engineering Intern}{Checkbook}{March 2021 - August 2021}
\begin{itemize}
\item Implemented and tested multiple functionalities end-to-end across 4+ repositories, including automating approvals for credit limit increase requests of \$5000+
\item Spearheaded engineering efforts to revamp entire React-based website resulting in reduced code duplication by 60\%
\item 7+ frameworks/languages used across internship

\smallskip
Stack:
\cvtag{Python}
\cvtag{Flask}
\cvtag{Angular}
\cvtag{React}
\cvtag{SQLAlchemy}

\end{itemize}

\divider

\cvevent{Backend Engineering Intern}{Alamere Software}{May 2020 - March 2021}
\begin{itemize}
\item Engineered data compression API to increase data storage efficiency of customer location data by 30\%
\item Designed web service to expose compressor API’s endpoints
\item Leveraged "RoaringBitmap", "GZip", and "LocationTech Spatial4j" compression libraries

\smallskip
Stack:
\cvtag{Java}
\cvtag{Spring Boot}
\cvtag{Springfox}
\cvtag{SQL}
% \cvtag{SQLAlchemy}

\end{itemize}

\medskip

\cvsection{Projects}

\cvproj{Bitfix}{Open-Source Project Connection Platform}{June 2020}{git.io/bitfix}
\begin{itemize}
    \item Connects volunteer coders to COVID-19 related open-source GitHub projects
    % \item Utilizes Gmail, Google Sheets, and Github API
    % \item Stores metadata in SQLite database, like each volunteer’s skill set and required skills for each project
    \item Grabs all open issues in given repositories, and matches each user to them based on overlapping skills
    \item Automatically emails users with more issues and updates spreadsheet with newly collected data
    
    \smallskip
    Stack:
    \cvtag{Python}
    \cvtag{SQLite}
    \cvtag{GitHub \& Gmail APIs}

\end{itemize}

\divider

\cvproj{Streaming Service Search}{Show/Movie Streaming Platform Launcher}{Sept 2020}{git.io/stream-search}
\begin{itemize}
    \item Automatically opens user-specified show/movie on any available streaming platform in browser
    % \item Allows user to cycle through all available streaming services through a command-line interface
    % \item Saves the user’s login information and other browser-related cookies
    \item Ranks user's streaming services from highest to lowest priority
    \item Operates on Mac, Windows, and Linux
    \item Supports over 15 streaming services, including Netflix, Hulu, Prime Video, Disney+, and more!

    \smallskip
    Stack:
    \cvtag{Python}

\end{itemize}

% \divider

% \cvevent{Gitlet}{Version Control System}{April 2020 - May 2020}
% \begin{itemize}
%     \item Git-like version control system that tracks files in a local repository
%     \item Commands to track files include init, add, commit, rm, log, branch, checkout, and more!
%     \item Wrote 1000+ lines of code in Java and spent 20+ hours to implement
% \end{itemize}

% \medskip

% use ONLY \newpage if you want to force a page break for
% ONLY the current column
% \newpage

% \cvsection{Publications}

% \nocite{*}

% \printbibliography[heading=pubtype,title={\printinfo{\faBook}{Books}},type=book]

% \divider

% \printbibliography[heading=pubtype,title={\printinfo{\faFile*[regular]}{Journal Articles}},type=article]

% \divider

% \printbibliography[heading=pubtype,title={\printinfo{\faUsers}{Conference Proceedings}},type=inproceedings]

%% Switch to the right column. This will now automatically move to the second
%% page if the content is too long.
\switchcolumn

% \cvsection{My Life Philosophy}

% \begin{quote}
% ``Something smart or heartfelt, preferably in one sentence.''
% \end{quote}


\cvsection{Languages}

\cvtag{Java}
\cvtag{Python}
\cvtag{Javascript}
\cvtag{SQL}
\linebreak
\cvtag{NoSQL}
\cvtag{HTML}
\cvtag{CSS}

\medskip

\cvsection{Frameworks/Tools}

\textbf{Backend:} 
\cvtag{Flask}
\cvtag{Spring Boot}
\cvtag{Springfox}
\cvtag{SQLAlchemy}
\cvtag{PostgreSQL}
\cvtag{MongoDB}
\divider

\textbf{Frontend:}
\cvtag{React}
\cvtag{Angular}
\cvtag{Gastby}

\divider

\textbf{Tools:}
\cvtag{Linux}
\cvtag{Git}
\cvtag{GitHub}
\cvtag{Docker}
\cvtag{Apache Maven}
\cvtag{RESTful API}
% \cvtag{Big Data}

% \cvsection{Most Proud of}


% \divider


% \divider



% \cvsection{Languages}

% \cvskill{English}{5}
% \divider

% \cvskill{Spanish}{4}
% \divider

% \cvskill{German}{3.5} %% Supports X.5 values.

%% Yeah I didn't spend too much time making all the
%% spacing consistent... sorry. Use \smallskip, \medskip,
%% \bigskip, \vspace etc to make adjustments.
\medskip

\cvsection{Education}

\cvgrad{\textbf{University of California, Berkeley}}{ Computer Science \kern 5pt | \kern 5pt GPA: 3.63}{August 2019 - May 2022}

\begin{itemize}
    \item Graduating one full year early
    \item Relevant coursework
\end{itemize}
\smallskip

\cvtag{Operating Systems}
\cvtag{Data Structures}

\cvtag{Algorithms}
\cvtag{Computer Security}

\cvtag{Intro to Artificial Intelligence}

\medskip

\cvsection{Hobbies}
\cvtag{Fantasy Novels}
\cvtag{Tennis}
\cvtag{Board Games}
\cvtag{Coding}
\cvtag{Video Games}
\cvtag{Music}

% \Large\color{LightBlue}\faBook
% \hspace{18pt}
% \Large\color{LightBlue}\faGamepad
% \hspace{18pt}
% \Large\color{LightBlue}\faUsers
% \hspace{18pt}
% \Large\color{LightBlue}\faHeadphones
% \hspace{18pt}
% \Large\color{LightBlue}\faBasketballBall
% \hspace{20pt}

% \divider



\end{paracol}


\end{document}
